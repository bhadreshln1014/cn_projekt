% ============================================================================
% Technical Documentation: LAN Multi-User Communication Application
% ============================================================================
% 
% This LaTeX document provides comprehensive technical documentation for a
% Python-based LAN communication application featuring:
%   - Video conferencing (640x480 @ 30fps, JPEG 60%)
%   - Audio communication (44.1kHz, server-side mixing, noise gate)
%   - Group chat (broadcast & private messaging)
%   - Screen sharing (960x540 @ 10fps, mutual exclusion)
%   - File sharing (in-memory, up to 100MB)
%
% Architecture:
%   - Client-Server with star topology
%   - TCP: Ports 5000 (control), 5003 (screen control), 5005 (files)
%   - UDP: Ports 5001 (video), 5002 (audio), 5004 (screen data)
%   - Multi-threaded server and client
%   - PyQt6 GUI with Material Design
%
% To compile: pdflatex technical_documentation.tex
% Screenshots: Replace placeholder figures with actual images
% ============================================================================

\documentclass[12pt]{article}
\usepackage{graphicx}
\usepackage{geometry}
\usepackage{fancyhdr}
\usepackage{hyperref}
\usepackage{listings}
\usepackage{xcolor}
\usepackage{amsmath}
\usepackage{titlesec}
\usepackage{enumitem}
\usepackage{float}
\usepackage{booktabs}
\usepackage{array}
\usepackage{multirow}

\geometry{a4paper, margin=1in}

% Fix for fancyhdr warning
\setlength{\headheight}{14.49998pt}
\addtolength{\topmargin}{-2.49998pt}

\definecolor{codegreen}{rgb}{0,0.6,0}
\definecolor{codegray}{rgb}{0.5,0.5,0.5}
\definecolor{codepurple}{rgb}{0.58,0,0.82}
\definecolor{backcolour}{rgb}{0.95,0.95,0.92}
\definecolor{linkcolor}{rgb}{0,0.4,0.6}

\lstdefinestyle{mystyle}{
    backgroundcolor=\color{backcolour},   
    commentstyle=\color{codegreen},
    keywordstyle=\color{magenta},
    numberstyle=\tiny\color{codegray},
    stringstyle=\color{codepurple},
    basicstyle=\ttfamily\footnotesize,
    breakatwhitespace=false,         
    breaklines=true,                 
    captionpos=b,                    
    keepspaces=true,                 
    numbers=left,                    
    numbersep=5pt,                  
    showspaces=false,                
    showstringspaces=false,
    showtabs=false,                  
    tabsize=2
}

\lstset{style=mystyle}

\hypersetup{
    colorlinks=true,
    linkcolor=linkcolor,
    filecolor=magenta,      
    urlcolor=cyan,
    pdftitle={LAN Multi-User Communication Application},
    pdfpagemode=FullScreen,
}

\pagestyle{fancy}
\fancyhf{}
\fancyhead[L]{LAN Multi-User Communication Application}
\fancyhead[R]{\today}
\fancyfoot[C]{\thepage}

\title{\textbf{Technical Documentation} \\ 
       \Large LAN Multi-User Communication Application}
\author{Computer Networks Project}
\date{\today}

\begin{document}

\maketitle
\thispagestyle{empty}
\newpage

\tableofcontents
\newpage

\section{Project Overview}

\subsection{Project Goals and Objectives}

The goal of this project is to develop a robust, standalone, and server-based multi-user communication application that operates \textbf{exclusively over a Local Area Network (LAN)}. This system provides a comprehensive suite of collaboration tools, enabling teams to communicate and share information in environments where internet access is unavailable, unreliable, or restricted.

\subsubsection{Primary Objectives}
\begin{enumerate}
    \item \textbf{LAN-Only Operation}: The application must function completely within a local area network without requiring internet connectivity
    \item \textbf{Comprehensive Collaboration}: Provide an all-in-one solution integrating video, audio, chat, screen sharing, and file transfer
    \item \textbf{Socket-Based Communication}: Implement all network communications using low-level socket programming (TCP and UDP)
    \item \textbf{Real-Time Performance}: Achieve low-latency communication suitable for real-time video and audio conferencing
    \item \textbf{User-Friendly Interface}: Deliver an intuitive graphical interface that consolidates all functionalities in a single window
\end{enumerate}

\subsubsection{Target Use Cases}
\begin{itemize}
    \item Corporate meetings in secure, air-gapped environments
    \item Educational institutions with restricted internet access
    \item Remote locations with unreliable internet connectivity
    \item Classified or sensitive communications requiring network isolation
    \item Development and testing environments
\end{itemize}

\section{Executive Summary}
This document provides comprehensive technical documentation for the LAN-Based Multi-User Communication Application. The system is a feature-rich, real-time collaboration platform designed for local area networks, supporting video conferencing, audio communication, group chat, screen sharing, and file sharing capabilities.

The application implements a client-server architecture using Python 3.8+, leveraging multiple network protocols (TCP and UDP) to provide reliable control messaging and low-latency media streaming. It supports up to 10 concurrent users and provides a modern, intuitive user interface built with PyQt6.

\subsection{Technology Stack}
\begin{itemize}
    \item \textbf{Programming Language}: Python 3.8+
    \item \textbf{GUI Framework}: PyQt6 (Qt 6.x)
    \item \textbf{Video Processing}: OpenCV (cv2) 4.5+
    \item \textbf{Audio Processing}: PyAudio 0.2.11+
    \item \textbf{Screen Capture}: MSS (Multi-Screen Shot) 6.1+
    \item \textbf{Numerical Computing}: NumPy 1.19+
    \item \textbf{Optional Icons}: QtAwesome 1.0+ (Material Design icons)
    \item \textbf{Networking}: Python socket, struct, pickle libraries
    \item \textbf{Threading}: Python threading, queue libraries
\end{itemize}

\subsection{Key Features}
\begin{itemize}
    \item \textbf{Video Conferencing}: Real-time video streaming at 640x480, 30 FPS with JPEG compression (60\% quality)
    \item \textbf{Audio Communication}: Low-latency audio at 44.1kHz with server-side mixing and client-side noise gate
    \item \textbf{Group Chat}: Broadcast messaging and private messaging with chat history and desktop notifications
    \item \textbf{Screen Sharing}: Single-presenter screen broadcasting at 960x540, 10 FPS with mutual exclusion
    \item \textbf{File Sharing}: In-memory file storage (up to 100MB per file) with upload, download, and delete operations
    \item \textbf{Modern UI}: Google Meet-inspired interface with collapsible panels, grid layouts, and Material Design icons
\end{itemize}

\subsection{Network Architecture and Socket Programming}

\subsubsection{LAN-Only Operation}

This application is specifically designed to operate \textbf{exclusively over Local Area Networks (LAN)} with \textbf{no internet connectivity required}:

\begin{itemize}
    \item \textbf{Zero Internet Dependency}: All communication occurs within the local network
    \item \textbf{Offline-First Design}: Application fully functional without any external network access
    \item \textbf{Network Isolation}: Suitable for air-gapped, secure, or restricted environments
    \item \textbf{Server Discovery}: Clients connect via LAN IP address (e.g., 192.168.1.100)
    \item \textbf{No Cloud Services}: No external servers, APIs, or cloud dependencies
\end{itemize}

\subsubsection{Socket Programming Implementation}

All network communication is implemented using \textbf{low-level TCP/IP socket programming}:

\textbf{TCP Sockets (Reliable, Connection-Oriented)}:
\begin{lstlisting}[language=Python, caption=TCP socket creation example]
# Server TCP socket for control messages
self.tcp_socket = socket.socket(socket.AF_INET, socket.SOCK_STREAM)
self.tcp_socket.setsockopt(socket.SOL_SOCKET, socket.SO_REUSEADDR, 1)
self.tcp_socket.bind((SERVER_HOST, SERVER_TCP_PORT))
self.tcp_socket.listen(MAX_USERS)

# Client TCP connection
self.tcp_socket = socket.socket(socket.AF_INET, socket.SOCK_STREAM)
self.tcp_socket.connect((server_ip, SERVER_TCP_PORT))
\end{lstlisting}

\textbf{UDP Sockets (Unreliable, Connectionless)}:
\begin{lstlisting}[language=Python, caption=UDP socket creation example]
# Server UDP socket for video streaming
self.udp_socket = socket.socket(socket.AF_INET, socket.SOCK_DGRAM)
self.udp_socket.setsockopt(socket.SOL_SOCKET, socket.SO_REUSEADDR, 1)
self.udp_socket.bind((SERVER_HOST, SERVER_UDP_PORT))

# Client UDP socket for sending video
self.udp_socket = socket.socket(socket.AF_INET, socket.SOCK_DGRAM)
self.udp_socket.sendto(frame_data, (server_address, SERVER_UDP_PORT))
\end{lstlisting}

\textbf{Protocol Usage}:
\begin{itemize}
    \item \textbf{TCP}: Control messages, chat, file transfers (ports 5000, 5003, 5005)
    \item \textbf{UDP}: Video, audio, screen sharing (ports 5001, 5002, 5004)
    \item \textbf{Binary Protocols}: struct.pack/unpack for efficient data serialization
    \item \textbf{Text Protocols}: UTF-8 encoded strings for control messages
\end{itemize}

\subsection{Core Functional Modules}

The application successfully implements all five required core modules using socket programming:

\subsubsection{Module 1: Multi-User Video Conferencing}

\textbf{Implementation Status}: Fully Implemented

\textbf{Key Features}:
\begin{itemize}
    \item \textbf{Video Capture}: OpenCV captures 640x480 frames from webcam at 30 FPS
    \item \textbf{Compression}: Real-time JPEG compression at 60\% quality (~50 KB per frame)
    \item \textbf{Transmission}: UDP socket on port 5001 for low-latency streaming
    \item \textbf{Server Broadcasting}: Server receives from all clients and broadcasts to all others
    \item \textbf{Client Rendering}: Dynamic grid layout (1x1 to 4x4) displays all participant videos simultaneously
    \item \textbf{Packet Format}: \texttt{[4 bytes: client\_id][remaining: JPEG frame data]}
\end{itemize}

\textbf{Protocol}: UDP for minimal latency, acceptable packet loss for real-time video

\begin{figure}[H]
    \centering
    \framebox{\parbox[c][4cm][c]{0.8\textwidth}{
        \centering \textit{[PLACEHOLDER: Video Grid Screenshot]} \\
        \vspace{0.5cm}
        \small{Shows multiple participant video feeds in grid layout}
    }}
    \caption{Multi-user video conferencing with grid layout}
    \label{fig:video_grid}
\end{figure}

\subsubsection{Module 2: Multi-User Audio Conferencing}

\textbf{Implementation Status}: Fully Implemented

\textbf{Key Features}:
\begin{itemize}
    \item \textbf{Audio Capture}: PyAudio captures 44.1kHz, 16-bit mono audio in 1024-sample chunks (~23ms)
    \item \textbf{Encoding}: Raw PCM format, 2048 bytes per chunk
    \item \textbf{Transmission}: UDP socket on port 5002 for real-time delivery
    \item \textbf{Server Mixing}: NumPy-based audio mixing - each client receives all others' audio (excluding own)
    \item \textbf{Noise Gate}: Client-side RMS threshold filter prevents echo and background noise
    \item \textbf{Playback}: Jitter buffer (20 frames max, 5 frame prebuffer) ensures smooth audio
\end{itemize}

\textbf{Protocol}: UDP with server-side mixing to prevent feedback loops

\subsubsection{Module 3: Slide \& Screen Sharing}

\textbf{Implementation Status}: Fully Implemented

\textbf{Key Features}:
\begin{itemize}
    \item \textbf{Presenter Role}: Mutual exclusion - only one presenter at a time
    \item \textbf{Screen Capture}: MSS library captures primary monitor at 960x540, 10 FPS
    \item \textbf{Compression}: JPEG at 50\% quality for balance of clarity and bandwidth (~30-40 KB per frame)
    \item \textbf{Content Transmission}: UDP socket on port 5004 for screen frames
    \item \textbf{Control Channel}: TCP socket on port 5003 for presenter permission management
    \item \textbf{Broadcasting}: Server relays presenter's screen to all other clients
    \item \textbf{UI Controls}: Start/stop presenting buttons with visual feedback
\end{itemize}

\textbf{Protocol}: TCP for control (reliability), UDP for screen data (low latency)

\begin{figure}[H]
    \centering
    \framebox{\parbox[c][4cm][c]{0.8\textwidth}{
        \centering \textit{[PLACEHOLDER: Screen Sharing in Progress]} \\
        \vspace{0.5cm}
        \small{Shows presenter's screen displayed in client interface}
    }}
    \caption{Screen sharing with active presenter}
    \label{fig:screen_share_active}
\end{figure}

\subsubsection{Module 4: Group Text Chat}

\textbf{Implementation Status}: Fully Implemented

\textbf{Key Features}:
\begin{itemize}
    \item \textbf{Message Transmission}: TCP socket on port 5000 for reliable delivery
    \item \textbf{Message Format}: \texttt{CHAT:<sender\_id>:<username>:<timestamp>:<message>}
    \item \textbf{Broadcasting}: Server receives and immediately broadcasts to all connected clients
    \item \textbf{Chat History}: Chronological log with sender username, timestamp, and message content
    \item \textbf{Private Messages}: Multi-recipient private messaging with \texttt{PRIVATE\_CHAT} protocol
    \item \textbf{Notifications}: Desktop popups for new messages when chat panel is closed
\end{itemize}

\textbf{Protocol}: TCP for guaranteed message delivery and ordering

\begin{figure}[H]
    \centering
    \framebox{\parbox[c][5cm][c]{0.65\textwidth}{
        \centering \textit{[PLACEHOLDER: Chat Panel with Message History]} \\
        \vspace{0.5cm}
        \small{Shows chat interface with chronological message log and input field}
    }}
    \caption{Group text chat with message history}
    \label{fig:chat_history}
\end{figure}

\subsubsection{Module 5: File Sharing}

\textbf{Implementation Status}: Fully Implemented

\textbf{Key Features}:
\begin{itemize}
    \item \textbf{File Selection}: Qt file dialog for local file selection
    \item \textbf{Upload Protocol}: TCP socket on port 5005 - \texttt{UPLOAD:<client\_id>:<filename>:<filesize>}
    \item \textbf{Chunked Transfer}: 8KB chunks for reliable transmission of large files
    \item \textbf{Server Storage}: In-memory file storage with metadata (filename, size, uploader, timestamp)
    \item \textbf{Distribution}: Server broadcasts \texttt{FILE\_OFFER} to all clients upon successful upload
    \item \textbf{Download}: Clients request files via \texttt{DOWNLOAD:<file\_id>} command
    \item \textbf{Progress Tracking}: Real-time progress bars for upload and download operations
    \item \textbf{File Management}: Uploaders can delete their own files; server broadcasts \texttt{FILE\_DELETED}
\end{itemize}

\textbf{Protocol}: TCP for error-free file delivery with progress tracking

\begin{figure}[H]
    \centering
    \framebox{\parbox[c][4cm][c]{0.65\textwidth}{
        \centering \textit{[PLACEHOLDER: File Transfer with Progress Bar]} \\
        \vspace{0.5cm}
        \small{Shows file list with upload/download progress indicators}
    }}
    \caption{File sharing with transfer progress}
    \label{fig:file_progress}
\end{figure}

\section{System Architecture}

\subsection{Overview}
The application follows a centralized client-server architecture with a star topology. All communication flows through a central server, which manages connections, routes data, and maintains session state. This architecture simplifies client implementation and provides centralized control over all application features.

\subsection{Network Topology}
Figure~\ref{fig:topology} illustrates the network topology where all clients connect to a central server.

\begin{figure}[H]
    \centering
    \small
    \begin{verbatim}
                        +-------------------------+
                        |   Central Server        |
                        |  (e.g., 192.168.1.100)  |
                        |                         |
                        |  TCP Ports: 5000, 5003, |
                        |             5005        |
                        |  UDP Ports: 5001, 5002, |
                        |             5004        |
                        +------------+------------+
                                     |
             +-----------------------+-----------------------+
             |                       |                       |
    +--------v--------+    +---------v--------+    +---------v--------+
    | Client 1 (Alice)|    | Client 2 (Bob)   |    | Client 3 (Charlie)|
    | ID: 0           |    | ID: 1            |    | ID: 2             |
    |                 |    |                  |    |                   |
    | TCP: Control,   |    | TCP: Control,    |    | TCP: Control,     |
    |      Screen,    |    |      Screen,     |    |      Screen,      |
    |      File       |    |      File        |    |      File         |
    | UDP: Video,     |    | UDP: Video,      |    | UDP: Video,       |
    |      Audio,     |    |      Audio,      |    |      Audio,       |
    |      Screen     |    |      Screen      |    |      Screen       |
    +-----------------+    +------------------+    +-------------------+
    \end{verbatim}
    \caption{Network topology showing server-client architecture with star topology}
    \label{fig:topology}
\end{figure}

All clients establish both TCP and UDP connections with the server. The server acts as a central hub, receiving media streams from clients and broadcasting them to all other connected clients.

\subsection{Server Architecture}

The server is a multi-threaded application that handles multiple concurrent operations:

\subsubsection{Server Components}
\begin{itemize}
    \item \textbf{Main Thread}: Accepts incoming TCP connections on port 5000 and spawns client handler threads
    \item \textbf{UDP Video Receiver Thread}: Receives video frames from clients via UDP port 5001 and broadcasts to all others
    \item \textbf{UDP Audio Receiver Thread}: Receives audio chunks via UDP port 5002, performs server-side mixing, and broadcasts mixed audio
    \item \textbf{Screen Sharing Control Thread}: Manages presenter permissions via TCP port 5003 using mutual exclusion
    \item \textbf{Screen UDP Receiver Thread}: Receives screen frames via UDP port 5004 and broadcasts to all clients
    \item \textbf{File Transfer Thread}: Handles file upload/download operations via TCP port 5005
    \item \textbf{Client Handler Threads}: One per connected client for TCP control communication (heartbeat, chat, user list updates)
\end{itemize}

The server uses thread locks (\texttt{clients\textunderscore lock}, \texttt{frames\textunderscore lock}, \texttt{audio\textunderscore lock}, \texttt{presenter\textunderscore lock}, \texttt{chat\textunderscore lock}, \texttt{files\textunderscore lock}) to ensure thread-safe access to shared data structures.

\subsubsection{Server Data Structures}
\begin{lstlisting}[language=Python, caption=Server data structures]
# Connected clients
clients = {
    client_id: {
        'tcp_conn': socket,
        'address': (ip, port),
        'udp_address': (ip, port),
        'audio_address': (ip, port),
        'screen_udp_address': (ip, port),
        'username': str
    }
}

# Video frames buffer
video_frames = {
    client_id: jpeg_frame_data
}

# Audio buffers
audio_buffers = {
    client_id: pcm_audio_data
}

# Chat history
chat_history = [
    {
        'client_id': id,
        'username': name,
        'message': text,
        'timestamp': time
    }
]

# Shared files
shared_files = {
    file_id: {
        'filename': name,
        'size': bytes,
        'uploader_id': id,
        'uploader_name': name,
        'data': bytes,
        'timestamp': time
    }
}
\end{lstlisting}

\subsection{Client Architecture}

The client application provides the user interface and manages media capture and playback:

\subsubsection{Client Components}
\begin{itemize}
    \item \textbf{Main/GUI Thread}: Renders the PyQt6 interface at ~30 FPS, handles user interactions, and updates video grid
    \item \textbf{Video Capture Thread}: Captures 640x480 frames from webcam using OpenCV, compresses to JPEG at 60\% quality
    \item \textbf{Video Sender Thread}: Sends compressed video frames via UDP to server (embedded in capture thread)
    \item \textbf{Video Receiver Thread}: Receives video frames from server via UDP and stores in \texttt{video\_streams} dict
    \item \textbf{Audio Capture Thread}: Captures audio from microphone (1024 samples/chunk), applies noise gate threshold (RMS > 100) to prevent echo
    \item \textbf{Audio Sender Thread}: Sends audio chunks via UDP to server (embedded in capture thread)
    \item \textbf{Audio Playback Worker Thread}: Plays received mixed audio with jitter buffer (queue size: 20, prebuffer: 5 frames)
    \item \textbf{TCP Control Receiver Thread}: Receives control messages (user list, chat, file offers, presenter status)
    \item \textbf{Screen Capture Thread}: Captures primary monitor using MSS library, compresses to JPEG at 50\% quality when presenting
    \item \textbf{Screen Receiver Thread}: Receives and displays shared screen frames via UDP
\end{itemize}

The client uses PyQt6 signals (\texttt{chat\textunderscore message\textunderscore received}, \texttt{user\textunderscore join\textunderscore signal}, \texttt{user\textunderscore left\textunderscore signal}, etc.) for thread-safe GUI updates and locks (\texttt{streams\textunderscore lock}, \texttt{users\textunderscore lock}, \texttt{screen\textunderscore lock}) for shared data.

\subsection{Configuration Management}

The application uses a centralized configuration file (\texttt{src/common/config.py}) that defines all network ports, media parameters, and application constants. This file is imported by both server and client, ensuring consistency:

\begin{lstlisting}[language=Python, caption=Configuration parameters (config.py)]
# Server Configuration
SERVER_HOST = '0.0.0.0'           # Listen on all interfaces
SERVER_TCP_PORT = 5000            # Control messages
SERVER_UDP_PORT = 5001            # Video streaming
SERVER_AUDIO_PORT = 5002          # Audio streaming
SERVER_SCREEN_PORT = 5003         # Screen control (TCP)
SERVER_SCREEN_UDP_PORT = 5004     # Screen data (UDP)
SERVER_FILE_PORT = 5005           # File transfer

# Video Configuration
VIDEO_WIDTH = 640
VIDEO_HEIGHT = 480
VIDEO_FPS = 30
VIDEO_QUALITY = 60                # JPEG compression (0-100)

# Audio Configuration
AUDIO_RATE = 44100                # Sample rate (Hz)
AUDIO_CHUNK = 1024                # Samples per chunk
AUDIO_CHANNELS = 1                # Mono audio
AUDIO_FORMAT = 8                  # pyaudio.paInt16
AUDIO_BUFFER_SIZE = 10            # Jitter buffer size

# Screen Sharing Configuration
SCREEN_WIDTH = 960
SCREEN_HEIGHT = 540
SCREEN_FPS = 10
SCREEN_QUALITY = 50
MAX_SCREEN_PACKET_SIZE = 65000

# Network Configuration
MAX_PACKET_SIZE = 65507           # Max UDP packet
CHUNK_SIZE = 60000                # Video chunk size

# Session Configuration
MAX_USERS = 10
HEARTBEAT_INTERVAL = 5            # seconds

# File Transfer Configuration
FILE_CHUNK_SIZE = 8192            # 8KB chunks
MAX_FILE_SIZE = 100 * 1024 * 1024 # 100MB limit
\end{lstlisting}

This centralized configuration allows easy tuning of performance parameters without modifying core application code.

\subsubsection{Client Data Structures}
\begin{lstlisting}[language=Python, caption=Client data structures]
# Video streams from other users
video_streams = {
    client_id: numpy_array_frame
}

# User list
users = {
    client_id: username
}

# Audio buffer for jitter compensation
audio_buffer = Queue(maxsize=20)
\end{lstlisting}

\section{Session Management}

The server implements robust session management to handle user connections, disconnections, and session lifecycle gracefully.

\subsection{Connection Establishment}

When a client connects to the server:

\begin{enumerate}
    \item \textbf{TCP Connection}: Client establishes TCP connection to server on port 5000
    \item \textbf{Authentication}: Client sends \texttt{CONNECT:<username>} message
    \item \textbf{ID Assignment}: Server assigns unique \texttt{client\_id} (sequential counter)
    \item \textbf{Registration}: Server stores client information in \texttt{clients} dictionary
    \item \textbf{Acknowledgment}: Server responds with \texttt{ID:<client\_id>}
    \item \textbf{User List Broadcast}: Server broadcasts updated user list to all connected clients
    \item \textbf{Thread Spawn}: Server creates dedicated handler thread for this client
\end{enumerate}

\begin{lstlisting}[language=Python, caption=Server connection handling]
def handle_client(self, conn, address):
    """Handle a client TCP connection"""
    client_id = None
    
    # Receive initial connection message with username
    data = conn.recv(1024).decode('utf-8')
    
    if data.startswith("CONNECT:"):
        username = data.split(":", 1)[1]
        
        with self.clients_lock:
            client_id = self.client_id_counter
            self.client_id_counter += 1
            
            self.clients[client_id] = {
                'tcp_conn': conn,
                'address': address,
                'udp_address': None,
                'username': username
            }
        
        # Send client ID and broadcast user list
        conn.send(f"ID:{client_id}".encode('utf-8'))
        self.broadcast_user_list()
\end{lstlisting}

\subsection{Heartbeat Mechanism}

To detect disconnected or unresponsive clients:

\begin{itemize}
    \item \textbf{Interval}: Every 5 seconds (configurable via \texttt{HEARTBEAT\_INTERVAL})
    \item \textbf{Protocol}: Client sends \texttt{PING}, server responds with \texttt{PONG}
    \item \textbf{Failure Detection}: If client doesn't respond, connection is marked as dead
    \item \textbf{Keep-Alive}: Prevents firewall/NAT timeout on idle connections
\end{itemize}

\subsection{Graceful Disconnection}

When a client disconnects (voluntarily or due to network failure), the server:

\begin{enumerate}
    \item Detects TCP connection close or timeout
    \item Removes client entry from \texttt{clients} dictionary
    \item Frees video frames, audio buffers, and other resources
    \item Releases presenter slot if client was presenting
    \item Broadcasts updated user list to remaining clients
    \item Sends user-left notifications to other clients
\end{enumerate}

Session continues uninterrupted for remaining users - no single client failure affects others.

\subsection{Dynamic Join/Leave Support}

\begin{itemize}
    \item \textbf{Late Join}: New users can join ongoing conferences and see all active participants immediately
    \item \textbf{Chat History}: Recent messages are available to new joiners
    \item \textbf{File Access}: Shared files remain accessible to new participants
    \item \textbf{No Interruption}: Existing participants experience no disruption
    \item \textbf{Auto-Recovery}: Users can reconnect after network issues
    \item \textbf{Maximum Capacity}: Up to 10 concurrent users (configurable)
\end{itemize}

\section{Communication Protocols}

The application uses multiple network protocols, each optimized for its specific purpose. TCP is used for reliable control messaging and file transfers, while UDP is used for real-time media streaming where low latency is critical.

\subsection{Port Allocation}

\begin{table}[H]
\centering
\begin{tabular}{|l|l|l|p{6cm}|}
\hline
\textbf{Port} & \textbf{Protocol} & \textbf{Purpose} & \textbf{Description} \\
\hline
5000 & TCP & Control Messages & User authentication, session management, chat \\
\hline
5001 & UDP & Video Streaming & Real-time video frame transmission \\
\hline
5002 & UDP & Audio Streaming & Real-time audio chunk transmission \\
\hline
5003 & TCP & Screen Control & Screen sharing permissions and control \\
\hline
5004 & UDP & Screen Data & Screen frame transmission \\
\hline
5005 & TCP & File Transfer & File upload and download \\
\hline
\end{tabular}
\caption{Port allocation for different services}
\label{tab:ports}
\end{table}

\subsection{Module 1: Video Conferencing Protocol}

\subsubsection{Connection Establishment}
When a client starts, it establishes a TCP connection to the server on port 5000:

\begin{lstlisting}[caption=Client connection message]
Client -> Server: "CONNECT:<username>"
Server -> Client: "ID:<client_id>"
Server -> All Clients: "USERS:<serialized_user_list>"
\end{lstlisting}

The serialized user list is a pickled Python list containing dictionaries with user information.

\subsubsection{Video Streaming}
Video frames are transmitted via UDP on port 5001. Each packet contains:
\begin{itemize}
    \item \textbf{Client ID} (4 bytes): Identifies the sender
    \item \textbf{JPEG Frame Data} (variable length): Compressed video frame
\end{itemize}

\textbf{Packet Format:}
\begin{verbatim}
+----------------+--------------------------------+
| Client ID      | JPEG Frame Data                |
| (4 bytes)      | (variable, ~50KB)              |
| uint32         | Raw bytes                      |
+----------------+--------------------------------+
\end{verbatim}

\textbf{Video Specifications:}
\begin{itemize}
    \item Resolution: 640x480 pixels
    \item Frame Rate: 30 FPS
    \item Compression: JPEG at 60\% quality
    \item Average Packet Size: 50 KB per frame
\end{itemize}

\begin{figure}[H]
    \centering
    \small
    \begin{verbatim}
Client A                      Server                     Client B, C
  |                             |                             |
  | [Capture Video]             |                             |
  | [Compress JPEG 60%]         |                             |
  |                             |                             |
  | UDP: [ID:0][JPEG_DATA]      |                             |
  |---------------------------->|                             |
  |                             | [Extract ID:0]              |
  |                             | [Store frame]               |
  |                             |                             |
  |                             | UDP: [ID:0][JPEG_DATA]      |
  |                             |---------------------------->|
  |                             |---------------------------->|
  |                             |                             | [Decode JPEG]
  |                             |                             | [Display Video]
  |                             |                             |
    \end{verbatim}
    \caption{Video conferencing data flow showing capture, transmission, and broadcast}
    \label{fig:video_flow}
\end{figure}

\subsubsection{Heartbeat Mechanism}
To detect disconnected clients, a heartbeat mechanism is implemented:
\begin{lstlisting}
Every 5 seconds:
    Client -> Server: "PING"
    Server -> Client: "PONG"
\end{lstlisting}

\subsection{Module 2: Audio Communication Protocol}

\subsubsection{Audio Specifications}
\begin{itemize}
    \item Sample Rate: 44,100 Hz
    \item Channels: Mono (1 channel)
    \item Format: 16-bit PCM
    \item Chunk Size: 1,024 samples (~23ms of audio)
    \item Transport: UDP on port 5002
\end{itemize}

\subsubsection{Audio Packet Format}
\begin{verbatim}
+----------------+--------------------------------+
| Client ID      | PCM Audio Data                 |
| (4 bytes)      | (2048 bytes: 1024 samples x 2) |
| uint32         | Raw bytes                      |
+----------------+--------------------------------+
\end{verbatim}

\subsubsection{Client-Side Noise Gate}
To prevent echo and reduce background noise, the client implements a noise gate that only sends audio above a certain threshold:

\begin{lstlisting}[language=Python, caption=Client-side noise gate implementation]
NOISE_GATE_THRESHOLD = 100  # RMS threshold

while self.audio_capturing and self.connected:
    # Read audio data
    audio_data = self.audio_stream_input.read(AUDIO_CHUNK)
    
    # Convert to numpy array
    audio_array = np.frombuffer(audio_data, dtype=np.int16)
    
    # Calculate RMS (loudness)
    rms = np.sqrt(np.mean(audio_array.astype(np.float32) ** 2))
    
    # Only send if above noise gate threshold
    if rms > NOISE_GATE_THRESHOLD:
        packet = struct.pack('I', self.client_id) + audio_data
        self.audio_udp_socket.sendto(packet, 
                                     (server_address, AUDIO_PORT))
    # else: too quiet, don't send (reduces echo/feedback)
\end{lstlisting}

This prevents the microphone from picking up audio playing from the speakers and sending it back, which would cause echo.

\subsubsection{Server-Side Audio Mixing}
The server implements audio mixing to prevent audio feedback and echo. Each client receives a mixed stream containing all other participants' audio except their own.

\begin{lstlisting}[language=Python, caption=Server audio mixing implementation]
def mix_and_broadcast_audio(self):
    """Mix all audio streams and broadcast to all clients"""
    with self.audio_lock:
        if len(self.audio_buffers) == 0:
            return
        
        # Convert all audio buffers to float32 numpy arrays
        audio_data_map = {}
        for client_id, audio_data in self.audio_buffers.items():
            audio_array = np.frombuffer(audio_data, dtype=np.int16)
            audio_data_map[client_id] = audio_array.astype(np.float32)
        
        # Broadcast to each client (excluding their own audio)
        with self.clients_lock:
            for target_client_id, client_info in self.clients.items():
                if client_info.get('audio_address') is not None:
                    # Collect audio from all EXCEPT target client
                    audio_arrays = []
                    for source_id, audio_array in audio_data_map.items():
                        if source_id != target_client_id:
                            audio_arrays.append(audio_array)
                    
                    if len(audio_arrays) > 0:
                        # Ensure all arrays same length
                        min_length = min(len(arr) for arr in audio_arrays)
                        audio_arrays = [arr[:min_length] for arr in audio_arrays]
                        
                        # Average all audio streams (prevents clipping)
                        mixed_audio = np.mean(audio_arrays, axis=0)
                        
                        # Convert back to int16 and send
                        mixed_audio = np.clip(mixed_audio, -32768, 32767)
                        mixed_data = mixed_audio.astype(np.int16).tobytes()
                        
                        self.audio_socket.sendto(mixed_data, 
                                                client_info['audio_address'])
\end{lstlisting}

Each client receives a mixed audio stream containing all other participants' audio, but not their own, preventing echo and feedback. The server performs mixing at regular intervals (every 23ms) matching the audio chunk duration.

\begin{figure}[H]
    \centering
    \small
    \begin{verbatim}
Client A              Server                Client B        Client C
  |                     |                      |               |
  | UDP: [ID:0][Audio]  |                      |               |
  |-------------------->|                      |               |
  |                     | [Store in buffer]    |               |
  |                     |                      |               |
  |                     | UDP: [ID:1][Audio]   |               |
  |                     |<---------------------|               |
  |                     | [Store in buffer]    |               |
  |                     |                      |               |
  |                     | [Mix audio from B,C] |               |
  |                     | (exclude A)          |               |
  |                     |                      |               |
  | UDP: [Mixed Audio]  |                      |               |
  |<--------------------|                      |               |
  | [Play]              |                      |               |
  |                     |                      |               |
  |                     | [Mix audio from A,C] |               |
  |                     | (exclude B)          |               |
  |                     |                      |               |
  |                     | UDP: [Mixed Audio]   |               |
  |                     |--------------------->|               |
  |                     |                      | [Play]        |
    \end{verbatim}
    \caption{Audio communication with server-side mixing to prevent echo}
    \label{fig:audio_flow}
\end{figure}

\subsection{Module 3: Screen Sharing Protocol}

\subsubsection{Presenter Management}
Screen sharing uses a mutual exclusion mechanism to ensure only one presenter at a time. The server maintains a \texttt{presenter\_id} variable protected by \texttt{presenter\_lock}:

\begin{lstlisting}[language=Python, caption=Server presenter permission control]
def handle_screen_share(self, conn, addr):
    """Handle screen sharing control from presenter"""
    # Receive client_id (4 bytes)
    client_id_data = conn.recv(4)
    client_id = struct.unpack('I', client_id_data)[0]
    
    # Check if this client can be presenter
    with self.presenter_lock:
        # Allow same client to reconnect
        # Deny if different presenter exists
        if (self.presenter_id is not None and 
            self.presenter_id != client_id):
            conn.sendall(b"DENIED")
            conn.close()
            return
        
        # Grant permission
        self.presenter_id = client_id
        conn.sendall(b"GRANTED")
    
    # Broadcast presenter status to all clients
    self.broadcast_presenter_status()
    
    # Keep connection alive for control
    # Connection close triggers presenter cleanup
    while self.running:
        data = conn.recv(1024)
        if not data or b"STOP" in data:
            break
    
    # Clean up on disconnect
    with self.presenter_lock:
        if self.presenter_id == client_id:
            self.presenter_id = None
            self.broadcast_presenter_status()
\end{lstlisting}

The TCP connection is kept alive to detect when the presenter disconnects or stops sharing, automatically releasing the presenter slot.

\subsubsection{Screen Capture Specifications}
\begin{itemize}
    \item Resolution: 960x540 pixels (16:9 aspect ratio)
    \item Frame Rate: 10 FPS (lower than video to reduce bandwidth)
    \item Compression: JPEG at 50\% quality
    \item Technology: MSS (Multi-Screen Shot) library
    \item Target Monitor: Primary monitor only
\end{itemize}

\subsubsection{Screen Frame Transmission}
Screen frames are sent via UDP on port 5004 with the same packet format as video:

\begin{verbatim}
+----------------+--------------------------------+
| Client ID      | JPEG Screen Frame Data         |
| (4 bytes)      | (variable, ~30-40KB)           |
| uint32         | Raw bytes                      |
+----------------+--------------------------------+
\end{verbatim}

\begin{figure}[H]
    \centering
    \small
    \begin{verbatim}
Presenter Client          Server                 Viewer Clients
     |                       |                         |
     | [Request to Present]  |                         |
     | TCP: Connect(5003)    |                         |
     |---------------------->|                         |
     | TCP: Send client_id   |                         |
     |---------------------->|                         |
     |                       | [Check presenter_id]    |
     |                       | if None OR same client: |
     |                       |   GRANT                 |
     |                       | else:                   |
     |                       |   DENY                  |
     |                       |                         |
     | TCP: "GRANTED"        |                         |
     |<----------------------|                         |
     |                       |                         |
     | [Start Screen Capture]|                         |
     | [Compress JPEG 50%]   |                         |
     |                       |                         |
     | UDP: [ID][Frame](5004)|                         |
     |---------------------->|                         |
     |                       | [Broadcast to all]      |
     |                       |                         |
     |                       | UDP: [ID][Frame]        |
     |                       |------------------------>|
     |                       |                         | [Display Screen]
     |                       |                         |
     | TCP: "STOP"           |                         |
     |---------------------->|                         |
     |                       | [Clear presenter_id]    |
     |                       | [Broadcast status]      |
    \end{verbatim}
    \caption{Screen sharing control flow with mutual exclusion}
    \label{fig:screen_flow}
\end{figure}

\subsection{Module 4: Group Chat Protocol}

The chat system supports both broadcast (group) and private messaging over the existing TCP control connection (port 5000).

\subsubsection{Message Types}

\textbf{1. Group Chat Messages}
\begin{lstlisting}
Client -> Server: "CHAT:<message>\n"
Server -> All Clients: "CHAT:<sender_id>:<username>:<timestamp>:<message>\n"

Example:
Client sends: "CHAT:Hello everyone\n"
Server broadcasts: "CHAT:0:Alice:14:30:15:Hello everyone\n"
\end{lstlisting}

\textbf{2. Private Chat Messages}
\begin{lstlisting}
Client -> Server: "PRIVATE_CHAT:<recipient_ids>:<message>\n"
Server -> Recipients: "PRIVATE_CHAT:<sender_id>|<username>|<timestamp>|<recipient_ids>|<message>\n"

Example:
Client sends: "PRIVATE_CHAT:2,3:Secret message\n"
Server sends: "PRIVATE_CHAT:0|Alice|14:30:15|2,3|Secret message\n"
\end{lstlisting}

\textbf{3. Chat History}
When a new client connects, the server sends recent chat history:
\begin{lstlisting}
Server -> New Client: "CHAT_HISTORY:<serialized_history>\n"
\end{lstlisting}

The serialized history is a pickled list of chat message dictionaries.

\begin{figure}[H]
    \centering
    \framebox{\parbox[c][5cm][c]{0.85\textwidth}{
        \centering \textit{[PLACEHOLDER: Chat Interface Screenshot]} \\
        \vspace{1cm}
        \small{Shows chat panel with message history, input field, and recipient selection}
    }}
    \caption{Chat interface showing group and private messages}
    \label{fig:chat_ui}
\end{figure}

\subsection{Module 5: File Sharing Protocol}

File sharing uses a dedicated TCP connection on port 5005 for reliable transfer.

\subsubsection{Upload Process}
\begin{lstlisting}[caption=File upload protocol]
1. Client -> Server (Port 5005 TCP): Connect
2. Client -> Server: "UPLOAD:<client_id>:<filename>:<filesize>\n"
3. Client -> Server: <file_data in 8KB chunks>
4. Server: Stores file in memory with assigned file_id
5. Server -> Client: "SUCCESS:<file_id>\n"
6. Server -> All Clients: 
   "FILE_OFFER:<file_id>:<filename>:<size>:<uploader_name>:<uploader_id>\n"
\end{lstlisting}

\subsubsection{Download Process}
\begin{lstlisting}[caption=File download protocol]
1. Client -> Server (Port 5005 TCP): Connect
2. Client -> Server: "DOWNLOAD:<file_id>\n"
3. Server: Looks up file in shared_files dictionary
4. Server -> Client: "FILE:<filename>:<filesize>\n"
5. Server -> Client: <file_data in 8KB chunks using sendall()>
6. Client: Saves to disk with QFileDialog selected location
\end{lstlisting}

\subsubsection{Delete Process}
Only the uploader can delete their own files:
\begin{lstlisting}[caption=File deletion with ownership verification]
1. Client -> Server (Port 5005 TCP): Connect
2. Client -> Server: "DELETE:<file_id>:<client_id>\n"
3. Server: Verifies uploader_id == client_id
4. If authorized:
   Server -> Client: "DELETE_SUCCESS:<file_id>\n"
   Server -> All Clients: "FILE_DELETED:<file_id>\n"
5. If not authorized:
   Server -> Client: "ERROR:Not authorized to delete this file\n"
\end{lstlisting}

\subsubsection{File Storage}
Files are stored in-memory on the server with a maximum size limit of 100 MB per file. The server maintains metadata including filename, size, uploader information, and timestamp.

\begin{figure}[H]
    \centering
    \framebox{\parbox[c][5cm][c]{0.85\textwidth}{
        \centering \textit{[PLACEHOLDER: File Sharing Panel Screenshot]} \\
        \vspace{1cm}
        \small{Shows file list with upload/download buttons and progress bars}
    }}
    \caption{File sharing interface}
    \label{fig:file_ui}
\end{figure}

\section{GUI Implementation}

The client application uses PyQt6 to provide a modern, responsive user interface inspired by Google Meet. \textbf{All five core functionalities are integrated into a single, unified window}, eliminating the need to switch between multiple applications.

\subsection{Single-Window Design Philosophy}

The application organizes all features into one cohesive interface:

\begin{itemize}
    \item \textbf{Central Video Grid}: Main viewing area for participant video feeds
    \item \textbf{Bottom Control Bar}: Always-visible buttons for core actions (mic, camera, screen share, leave)
    \item \textbf{Right Side Panels}: Collapsible panels for chat, files, people, and settings
    \item \textbf{Integrated Notifications}: Desktop popups overlay on main window
    \item \textbf{Contextual Controls}: Features accessible without leaving main window
\end{itemize}

\textbf{Design Benefits}:
\begin{itemize}
    \item Reduced cognitive load - users don't need to manage multiple windows
    \item Unified state management - all features aware of session context
    \item Efficient screen usage - maximizes video display area
    \item Consistent user experience - familiar control patterns across features
\end{itemize}

\begin{figure}[H]
    \centering
    \framebox{\parbox[c][6cm][c]{0.9\textwidth}{
        \centering \textit{[PLACEHOLDER: Complete UI Screenshot]} \\
        \vspace{1cm}
        \small{Shows unified interface with video grid, control bar, and open chat panel demonstrating all features in single window}
    }}
    \caption{Complete user interface showing all features integrated in single window}
    \label{fig:unified_ui}
\end{figure}

\subsection{Main Window Architecture}

The main window (\texttt{VideoConferenceClient}) inherits from \texttt{QMainWindow} and uses PyQt6 signals for thread-safe GUI updates:

\begin{lstlisting}[language=Python, caption=PyQt6 signals for thread-safe updates]
class VideoConferenceClient(QMainWindow):
    # Signals for thread-safe GUI updates
    chat_message_received = pyqtSignal(int, str, str, str)
    user_join_signal = pyqtSignal(str)
    user_left_signal = pyqtSignal(str)
    screen_share_denied_signal = pyqtSignal(str)
    # ... other signals
\end{lstlisting}

\subsection{Layout System}

The video grid supports multiple layout modes for flexible viewing:

\subsubsection{Tiled Layout (Grid Mode)}
\begin{itemize}
    \item \textbf{Auto Mode}: Automatically selects optimal grid size (1x1, 2x2, 3x3, or 4x4) based on participant count
    \item \textbf{Manual Modes}: User can force specific grid layouts (1x1, 2x2, 3x3, 4x4)
    \item \textbf{Dynamic Resizing}: Grid automatically adjusts when users join or leave
    \item \textbf{Equal Sizing}: All participant video tiles have equal dimensions
    \item \textbf{Default Mode}: Tiled layout is the default view on startup
\end{itemize}

\subsubsection{Spotlight Layout (Focus Mode)}
\begin{itemize}
    \item \textbf{Dedicated Display}: Large central area for featured content (screen share or main speaker)
    \item \textbf{Side Thumbnails}: Vertical strip showing other participant thumbnails
    \item \textbf{Automatic Activation}: Spotlight mode activates when screen sharing begins
    \item \textbf{Seamless Switching}: Returns to tiled mode when screen sharing ends
    \item \textbf{Screen Share Priority}: Screen share always displayed in spotlight main area
\end{itemize}

\textbf{Layout Selection}: Users can configure their preferred layout mode in the Settings panel. The layout adapts dynamically based on the number of active participants and whether screen sharing is active.

\subsection{Side Panels}

The interface features collapsible side panels:
\begin{itemize}
    \item \textbf{Chat Panel}: Message history with broadcast and private messaging
    \item \textbf{People Panel}: Participant list with selection for private messaging
    \item \textbf{Files Panel}: Shared files with upload/download functionality
    \item \textbf{Settings Panel}: Audio/video device selection and layout configuration
\end{itemize}

\subsection{Notification System}

Desktop notifications appear in the bottom-right corner for new chat messages, user joins/leaves, file uploads, and screen sharing changes. Notifications auto-dismiss after 4-5 seconds.

\subsection{Video Rendering}

Video frames are converted from OpenCV BGR format to Qt RGB format using \texttt{QImage} and \texttt{QPixmap}. The main thread updates video labels at approximately 30 FPS using a QTimer.

\section{User Guide}

This section provides step-by-step instructions for installing, configuring, and using the application.

\subsection{System Requirements}

\subsubsection{Software Requirements}
\begin{itemize}
    \item Python 3.8 or higher
    \item Operating System: Windows 10/11, Linux (Ubuntu 20.04+), or macOS 10.14+
    \item Webcam (built-in or external)
    \item Microphone and speakers/headphones
    \item Network: LAN connection or WiFi on the same network
\end{itemize}

\subsubsection{Python Dependencies}
\begin{lstlisting}[language=bash, caption=requirements.txt]
opencv-python>=4.5.0
numpy>=1.19.0
PyQt6>=6.0.0
pyaudio>=0.2.11
mss>=6.1.0
qtawesome>=1.0.0
\end{lstlisting}

\subsection{Installation}

\subsubsection{Step 1: Obtain the Project}
Download or clone the project repository to your computer.

\subsubsection{Step 2: Install Python}
Ensure Python 3.8 or higher is installed. Download from \url{https://www.python.org/downloads/} if needed.

On Windows, make sure to check "Add Python to PATH" during installation.

\subsubsection{Step 3: Install Dependencies}
Open a terminal/command prompt in the project directory and run:

\textbf{Windows:}
\begin{lstlisting}[language=bash]
pip install -r requirements.txt
\end{lstlisting}

\textbf{Linux/macOS:}
\begin{lstlisting}[language=bash]
pip3 install -r requirements.txt

# On Linux, you may also need:
sudo apt-get install python3-tk
sudo apt-get install portaudio19-dev  # for PyAudio
\end{lstlisting}

\subsubsection{Step 4: Verify Installation}
\begin{lstlisting}[language=bash]
python -c "import cv2, numpy, PyQt6, pyaudio, mss; print('All dependencies installed successfully!')"
\end{lstlisting}

\subsection{Starting the Application}

\subsubsection{Running the Server}

\textbf{Step 1: Find Server IP Address}

On the computer that will act as the server, find its local IP address:

\textbf{Windows:}
\begin{lstlisting}[language=bash]
ipconfig
# Look for "IPv4 Address" under your active network adapter
# Example: 192.168.1.100
\end{lstlisting}

\textbf{Linux/macOS:}
\begin{lstlisting}[language=bash]
ifconfig
# or
ip addr
# Look for inet address under your active network interface
\end{lstlisting}

\textbf{Step 2: Start the Server}

\textbf{Using the startup script (recommended):}
\begin{lstlisting}[language=bash]
# Windows
start_server.bat

# Linux/macOS
chmod +x start_server.sh
./start_server.sh
\end{lstlisting}

\textbf{Or run manually:}
\begin{lstlisting}[language=bash]
python src/server/server.py
\end{lstlisting}

You should see output similar to:
\begin{lstlisting}
[14:30:15] Server started
[14:30:15] TCP Control Port: 5000
[14:30:15] UDP Video Port: 5001
[14:30:15] UDP Audio Port: 5002
[14:30:15] TCP Screen Sharing Control Port: 5003
[14:30:15] UDP Screen Sharing Data Port: 5004
[14:30:15] TCP File Transfer Port: 5005
[14:30:15] Waiting for connections...
\end{lstlisting}

\begin{figure}[H]
    \centering
    \framebox{\parbox[c][4cm][c]{0.85\textwidth}{
        \centering \textit{[PLACEHOLDER: Server Console Screenshot]} \\
        \vspace{1cm}
        \small{Shows server startup messages and active connections}
    }}
    \caption{Server console output}
    \label{fig:server_console}
\end{figure}

\subsubsection{Running the Client}

On each computer that will participate (can include the server computer):

\textbf{Step 1: Start the Client Application}
\begin{lstlisting}[language=bash]
# Windows
start_client.bat

# Linux/macOS
chmod +x start_client.sh
./start_client.sh

# Or manually
python src/client/client.py
\end{lstlisting}

\textbf{Step 2: Connect to Server}

A connection dialog will appear (Figure~\ref{fig:connect_dialog}).

\begin{figure}[H]
    \centering
    \framebox{\parbox[c][5cm][c]{0.7\textwidth}{
        \centering \textit{[PLACEHOLDER: Connection Dialog Screenshot]} \\
        \vspace{1cm}
        \small{Shows server IP input, username input, and Connect button}
    }}
    \caption{Client connection dialog}
    \label{fig:connect_dialog}
\end{figure}

Enter the following information:
\begin{itemize}
    \item \textbf{Server IP}: The IP address of the server (e.g., 192.168.1.100)
    \begin{itemize}
        \item Use \texttt{127.0.0.1} or \texttt{localhost} if running on the same computer as the server
    \end{itemize}
    \item \textbf{Username}: Your desired username (must be unique)
\end{itemize}

Click the "Connect" button.

\subsection{Using the Application}

Once connected, the main application window opens (Figure~\ref{fig:main_window}).

\begin{figure}[H]
    \centering
    \framebox{\parbox[c][7cm][c]{0.9\textwidth}{
        \centering \textit{[PLACEHOLDER: Main Application Window Screenshot]} \\
        \vspace{1cm}
        \small{Shows video grid, control bar at bottom, and side panels}
    }}
    \caption{Main application interface}
    \label{fig:main_window}
\end{figure}

\subsubsection{Video Grid Layout}

The main viewing area displays video feeds in a grid layout:
\begin{itemize}
    \item \textbf{Auto Layout}: Automatically adjusts grid size (1x1, 2x2, 3x3, etc.) based on number of participants
    \item \textbf{Manual Layout}: Select fixed grid size from the settings menu
    \item \textbf{Self Video}: Your own video feed can be toggled on/off
    \item \textbf{Username Labels}: Each video tile displays the participant's username
\end{itemize}

\subsubsection{Control Bar}

The control bar at the bottom provides quick access to features:

\begin{table}[H]
\centering
\begin{tabular}{|l|p{10cm}|}
\hline
\textbf{Button} & \textbf{Function} \\
\hline
Camera & Toggle your webcam on/off \\
\hline
Microphone & Mute/unmute your microphone \\
\hline
Speaker & Mute/unmute audio output \\
\hline
Chat & Open/close the chat panel \\
\hline
People & Open/close the participants panel \\
\hline
Files & Open/close the file sharing panel \\
\hline
Present Screen & Start/stop screen sharing \\
\hline
Settings & Open audio/video device settings \\
\hline
Leave & Disconnect from the conference \\
\hline
\end{tabular}
\caption{Control bar buttons}
\label{tab:controls}
\end{table}

\begin{figure}[H]
    \centering
    \framebox{\parbox[c][2cm][c]{0.85\textwidth}{
        \centering \textit{[PLACEHOLDER: Control Bar Screenshot]} \\
        \vspace{0.5cm}
        \small{Shows all control buttons with icons}
    }}
    \caption{Control bar with feature buttons}
    \label{fig:control_bar}
\end{figure}

\subsubsection{Chat Feature}

\textbf{Opening Chat:}
Click the "Chat" button in the control bar to open the chat panel on the right side.

\textbf{Sending Messages:}
\begin{enumerate}
    \item Type your message in the text input field at the bottom of the chat panel
    \item Press Enter or click the Send button
    \item Your message appears in the chat history with your username and timestamp
\end{enumerate}

\textbf{Private Messaging:}
\begin{enumerate}
    \item Open the "People" panel
    \item Select one or more recipients by clicking their names
    \item Type your message in the chat input
    \item Your message is sent only to the selected recipients
    \item Private messages are indicated with a lock icon
\end{enumerate}

\textbf{Chat Features:}
\begin{itemize}
    \item Message history persists during your session
    \item New users receive recent chat history upon joining
    \item Timestamps are displayed for each message
    \item System messages notify you of user joins/leaves
    \item Desktop notifications for new messages (when chat panel is closed)
\end{itemize}

\begin{figure}[H]
    \centering
    \framebox{\parbox[c][6cm][c]{0.7\textwidth}{
        \centering \textit{[PLACEHOLDER: Chat Panel Screenshot]} \\
        \vspace{1cm}
        \small{Shows chat history with timestamps, message input, and send button}
    }}
    \caption{Chat panel interface}
    \label{fig:chat_panel}
\end{figure}

\subsubsection{Screen Sharing}

\textbf{Starting Screen Share:}
\begin{enumerate}
    \item Click the "Present Screen" button in the control bar
    \item \textbf{Mutual Exclusion Check}: Server verifies no other user is currently presenting
    \item \textbf{If someone is already presenting}:
    \begin{itemize}
        \item Warning dialog appears: "\textit{[Presenter Name] is currently presenting}"
        \item Informative message: "Only one user can share their screen at a time. Please wait until they finish."
        \item Screen share button automatically reverts to unchecked state
        \item You cannot start presenting until the current presenter stops
    \end{itemize}
    \item \textbf{If no one is presenting}:
    \begin{itemize}
        \item Your screen share starts immediately
        \item Your video feed is replaced by your screen in other participants' views
        \item \textbf{Layout switches to Spotlight mode} for all participants
        \item Button changes to "Stop Presenting" with red indicator
        \item Desktop notification sent to all other participants
    \end{itemize}
\end{enumerate}

\textbf{Stopping Screen Share:}
\begin{enumerate}
    \item Click the "Stop Presenting" button
    \item Your video feed resumes in other participants' views
    \item \textbf{Layout returns to Tiled mode} (or user's preferred layout)
    \item Server releases presenter slot, allowing others to present
    \item Desktop notification sent to all participants
\end{enumerate}

\textbf{Viewing Shared Screen:}
\begin{itemize}
    \item When another participant shares their screen, it replaces their video feed in your grid
    \item \textbf{Automatic layout switch}: Interface changes to Spotlight mode with screen in large main area
    \item The username label indicates who is presenting
    \item Screen updates at 10 FPS for smooth viewing (960x540 resolution, JPEG 50\% quality)
    \item \textbf{Desktop notification}: "User X started screen sharing"
    \item You can still use chat and other features while viewing a shared screen
    \item Other participant videos appear as thumbnails in the side panel (Spotlight mode)
\end{itemize}

\begin{figure}[H]
    \centering
    \framebox{\parbox[c][6cm][c]{0.85\textwidth}{
        \centering \textit{[PLACEHOLDER: Screen Sharing View Screenshot]} \\
        \vspace{1cm}
        \small{Shows a participant's shared screen displayed in the video grid}
    }}
    \caption{Screen sharing in progress}
    \label{fig:screen_share}
\end{figure}

\subsubsection{File Sharing}

\textbf{Uploading Files:}
\begin{enumerate}
    \item Click the "Files" button to open the file sharing panel
    \item Click the "Upload File" button
    \item Select a file from your computer (max 100 MB)
    \item A progress bar shows the upload progress
    \item Once uploaded, the file appears in everyone's file list
\end{enumerate}

\textbf{Downloading Files:}
\begin{enumerate}
    \item Open the file sharing panel
    \item Click the "Download" button next to the desired file
    \item Select where to save the file on your computer
    \item A progress bar shows the download progress
    \item The file is saved to your selected location
\end{enumerate}

\textbf{Deleting Files:}
\begin{itemize}
    \item You can only delete files that you uploaded
    \item Click the "Delete" button next to your file
    \item The file is removed from all participants' lists
\end{itemize}

\textbf{File Information Displayed:}
\begin{itemize}
    \item Filename
    \item File size (formatted as KB/MB)
    \item Uploader's name
    \item Upload timestamp
\end{itemize}

\begin{figure}[H]
    \centering
    \framebox{\parbox[c][6cm][c]{0.7\textwidth}{
        \centering \textit{[PLACEHOLDER: File Sharing Panel Screenshot]} \\
        \vspace{1cm}
        \small{Shows file list with upload/download/delete buttons and progress bars}
    }}
    \caption{File sharing panel}
    \label{fig:file_panel}
\end{figure}

\subsubsection{Audio/Video Settings}

\textbf{Accessing Settings:}
Click the "Settings" button in the control bar to open the settings panel.

\textbf{Available Settings:}
\begin{itemize}
    \item \textbf{Microphone Selection}: Choose your audio input device from available system microphones
    \item \textbf{Speaker Selection}: Choose your audio output device from available system speakers/headphones
    \item \textbf{Camera Selection}: Choose your video input device (if multiple cameras available)
    \item \textbf{Layout Mode}: Select grid layout preference:
    \begin{itemize}
        \item \textit{Auto}: Automatically adjusts grid size based on participant count
        \item \textit{1x1}: Single large video tile
        \item \textit{2x2}: Four video tiles in 2x2 grid
        \item \textit{3x3}: Nine video tiles in 3x3 grid
        \item \textit{4x4}: Sixteen video tiles in 4x4 grid
        \item \textit{Spotlight}: Large main area with side thumbnails (auto-activates during screen sharing)
    \end{itemize}
    \item \textbf{Show Self Video}: Toggle visibility of your own video in the grid
\end{itemize}

\textbf{Device Enumeration}: The application automatically detects all available audio/video devices on your system using PyAudio and OpenCV device enumeration. Changes to device selection take effect immediately without requiring a restart.

Changes to audio/video devices take effect immediately.

\begin{figure}[H]
    \centering
    \framebox{\parbox[c][5cm][c]{0.7\textwidth}{
        \centering \textit{[PLACEHOLDER: Settings Panel Screenshot]} \\
        \vspace{1cm}
        \small{Shows audio/video device dropdowns and layout options}
    }}
    \caption{Settings panel}
    \label{fig:settings}
\end{figure}

\subsubsection{People Panel}

The People panel shows all connected participants:
\begin{itemize}
    \item List of all usernames
    \item Selection checkboxes for private messaging
    \item Online status indicators
    \item Participant count
\end{itemize}

Click on names to select recipients for private messages.

\begin{figure}[H]
    \centering
    \framebox{\parbox[c][5cm][c]{0.6\textwidth}{
        \centering \textit{[PLACEHOLDER: People Panel Screenshot]} \\
        \vspace{1cm}
        \small{Shows list of participants with selection checkboxes}
    }}
    \caption{People panel showing participants}
    \label{fig:people_panel}
\end{figure}

\subsection{Disconnecting}

To leave the conference:
\begin{enumerate}
    \item Click the "Leave" button in the control bar, or
    \item Close the application window
    \item The server automatically detects the disconnection
    \item Other participants are notified of your departure
    \item Your video feed is removed from their grids
\end{enumerate}

\section{Troubleshooting}

\subsection{Connection Issues}

\textbf{Problem}: Cannot connect to server

\textbf{Solutions}:
\begin{itemize}
    \item Verify the server is running and showing "Waiting for connections..."
    \item Confirm the server IP address is correct
    \item Ensure both computers are on the same network
    \item Check firewall settings (allow ports 5000-5005)
    \item Try pinging the server: \texttt{ping <server\_ip>}
\end{itemize}

\subsection{Camera Issues}

\textbf{Problem}: Webcam not working

\textbf{Solutions}:
\begin{itemize}
    \item Check camera permissions in your operating system
    \item Ensure no other application is using the camera
    \item Try selecting a different camera in Settings
    \item Restart the application
    \item On Linux, verify you have proper permissions: \texttt{sudo usermod -aG video \$USER}
\end{itemize}

\subsection{Audio Issues}

\textbf{Problem}: No audio or echo/feedback

\textbf{Solutions}:
\begin{itemize}
    \item Check microphone and speaker are properly connected
    \item Verify audio devices in Settings panel
    \item Use headphones to prevent echo/feedback
    \item Check system audio settings and levels
    \item Ensure PyAudio is properly installed
    \item On Linux, you may need to install: \texttt{sudo apt-get install portaudio19-dev}
\end{itemize}

\subsection{Screen Sharing Issues}

\textbf{Problem}: Cannot start screen sharing

\textbf{Solutions}:
\begin{itemize}
    \item Only one person can present at a time - wait for current presenter to stop
    \item Check screen sharing permissions on macOS (System Preferences > Security \& Privacy > Screen Recording)
    \item Restart the application if the presenter slot is stuck
\end{itemize}

\subsection{File Transfer Issues}

\textbf{Problem}: File upload/download fails

\textbf{Solutions}:
\begin{itemize}
    \item Ensure file is under 100 MB limit
    \item Check available disk space
    \item Verify network connection is stable
    \item Check file permissions (read for upload, write for download location)
\end{itemize}

\subsection{Performance Issues}

\textbf{Problem}: Laggy video or choppy audio

\textbf{Solutions}:
\begin{itemize}
    \item Ensure good network connection (use wired Ethernet if possible)
    \item Close other bandwidth-intensive applications
    \item Reduce number of participants if possible
    \item Check CPU usage - close other applications
    \item Lower video quality (modify \texttt{VIDEO\_QUALITY} in \texttt{config.py})
\end{itemize}

\section{Technical Specifications Summary}

\subsection{Network Specifications}
\begin{table}[H]
\centering
\begin{tabular}{|l|l|}
\hline
\textbf{Parameter} & \textbf{Value} \\
\hline
Maximum Users & 10 concurrent users \\
\hline
Video Bandwidth & ~12 Mbps per user (640x480 @ 30fps, 60\% JPEG) \\
\hline
Audio Bandwidth & ~0.7 Mbps per user (44.1kHz, 16-bit mono) \\
\hline
Screen Bandwidth & ~2-4 Mbps (960x540 @ 10fps, 50\% JPEG) \\
\hline
Control Overhead & < 100 Kbps \\
\hline
Heartbeat Interval & 5 seconds \\
\hline
\end{tabular}
\caption{Network specifications}
\label{tab:network_specs}
\end{table}

\subsection{Media Specifications}
\begin{table}[H]
\centering
\begin{tabular}{|l|l|l|}
\hline
\textbf{Media Type} & \textbf{Parameter} & \textbf{Value} \\
\hline
\multirow{4}{*}{Video} & Resolution & 640x480 pixels \\
& Frame Rate & 30 FPS \\
& Compression & JPEG 60\% \\
& Average Frame Size & 50 KB \\
\hline
\multirow{4}{*}{Audio} & Sample Rate & 44,100 Hz \\
& Channels & Mono (1) \\
& Bit Depth & 16-bit PCM \\
& Chunk Size & 1,024 samples (23ms) \\
\hline
\multirow{4}{*}{Screen} & Resolution & 960x540 pixels \\
& Frame Rate & 10 FPS \\
& Compression & JPEG 50\% \\
& Average Frame Size & 30-40 KB \\
\hline
\end{tabular}
\caption{Media specifications}
\label{tab:media_specs}
\end{table}

\section{Platform Support and Compatibility}

\subsection{Supported Operating Systems}

The application is designed to be cross-platform and has been tested on:

\begin{table}[H]
\centering
\begin{tabular}{|l|l|l|}
\hline
\textbf{Operating System} & \textbf{Version} & \textbf{Status} \\
\hline
Windows & 10, 11 & Fully Supported \\
\hline
Linux & Ubuntu 20.04+, Debian 11+ & Fully Supported \\
\hline
macOS & 10.14 (Mojave) or higher & Fully Supported \\
\hline
\end{tabular}
\caption{Platform compatibility}
\label{tab:platforms}
\end{table}

\subsection{Platform-Specific Considerations}

\subsubsection{Windows}
\begin{itemize}
    \item \textbf{Python Installation}: Ensure "Add Python to PATH" is checked during installation
    \item \textbf{Firewall}: May need to allow Python through Windows Firewall for ports 5000-5005
    \item \textbf{Audio}: PyAudio typically works out-of-the-box with Windows audio drivers
    \item \textbf{Executables}: Can be packaged using PyInstaller for standalone \texttt{.exe} files
\end{itemize}

\subsubsection{Linux}
\begin{itemize}
    \item \textbf{Dependencies}: May need to install \texttt{python3-tk}, \texttt{portaudio19-dev}, and \texttt{python3-dev}
    \item \textbf{Permissions}: Camera/microphone access may require user in \texttt{video} and \texttt{audio} groups
    \item \textbf{Package Manager}: Use \texttt{apt}, \texttt{dnf}, or \texttt{pacman} depending on distribution
    \item \textbf{Executables}: Can be packaged using PyInstaller for standalone binaries
\end{itemize}

\subsubsection{macOS}
\begin{itemize}
    \item \textbf{Python Version}: Use Python 3.8+ from python.org or Homebrew
    \item \textbf{Permissions}: Requires granting Camera, Microphone, and Screen Recording permissions in System Preferences > Security \& Privacy
    \item \textbf{PyAudio Installation}: May require \texttt{brew install portaudio} before \texttt{pip install pyaudio}
    \item \textbf{Executables}: Can be packaged as \texttt{.app} bundles using PyInstaller
\end{itemize}

\subsection{Network Requirements}

\begin{itemize}
    \item \textbf{LAN Connectivity}: All devices must be on the same local network
    \item \textbf{No Internet Required}: Application functions completely offline
    \item \textbf{IP Configuration}: Static IP for server recommended; DHCP acceptable if IP noted
    \item \textbf{Firewall Rules}: Ports 5000-5005 must be accessible between client and server
    \item \textbf{Bandwidth}: Minimum 15 Mbps per user recommended for optimal experience
    \item \textbf{Latency}: LAN latency typically <10ms; wireless acceptable but wired Ethernet preferred
\end{itemize}

\section{Project Deliverables}

This project includes the following complete deliverables:

\subsection{1. Server Application}

\textbf{File}: \texttt{src/server/server.py}

\textbf{Description}: A standalone Python application that can be executed on any machine in the LAN to host the communication session.

\textbf{Features}:
\begin{itemize}
    \item Accepts connections from up to 10 clients simultaneously
    \item Manages all media stream broadcasting (video, audio, screen)
    \item Handles chat message relay and file storage
    \item Implements session management with graceful join/leave
    \item Provides console logging of all server activities
\end{itemize}

\textbf{Execution}:
\begin{lstlisting}[language=bash]
python src/server/server.py
# Or use provided scripts: start_server.bat / start_server.sh
\end{lstlisting}

\textbf{Packaging}: Can be compiled to standalone executable using PyInstaller:
\begin{lstlisting}[language=bash]
pyinstaller build_server.spec
# Generates: dist/server.exe (Windows) or dist/server (Linux/macOS)
\end{lstlisting}

\subsection{2. Client Application}

\textbf{File}: \texttt{src/client/client.py}

\textbf{Description}: A standalone Python application with PyQt6 GUI that users run to connect to the server and participate in the session.

\textbf{Features}:
\begin{itemize}
    \item Modern, intuitive graphical user interface
    \item All five core modules integrated in single window
    \item Connection dialog for server IP and username entry
    \item Real-time video grid showing all participants
    \item Audio/video device selection and configuration
    \item Chat, file sharing, and screen sharing controls
\end{itemize}

\textbf{Execution}:
\begin{lstlisting}[language=bash]
python src/client/client.py
# Or use provided scripts: start_client.bat / start_client.sh
\end{lstlisting}

\textbf{Packaging}: Can be compiled to standalone executable using PyInstaller:
\begin{lstlisting}[language=bash]
pyinstaller build_client.spec
# Generates: dist/client.exe (Windows) or dist/client (Linux/macOS)
\end{lstlisting}

\subsection{3. Source Code}

\textbf{Files}: Complete source code in \texttt{src/} directory

\textbf{Structure}:
\begin{verbatim}
src/
|-- client/
|   |-- __init__.py
|   +-- client.py          # Client application (4172 lines)
|-- server/
|   |-- __init__.py
|   +-- server.py          # Server application (934 lines)
+-- common/
    |-- __init__.py
    +-- config.py          # Shared configuration
\end{verbatim}

\textbf{Code Quality}:
\begin{itemize}
    \item \textbf{Fully Commented}: Comprehensive docstrings and inline comments
    \item \textbf{Type Hints}: Function signatures include parameter and return types
    \item \textbf{Error Handling}: Try-except blocks for robust error management
    \item \textbf{Modular Design}: Clear separation of concerns and responsibilities
    \item \textbf{Thread-Safe}: Proper use of locks for concurrent data access
\end{itemize}

\subsection{4. Configuration Files}

\textbf{Build Scripts}:
\begin{itemize}
    \item \texttt{build\_server.spec}: PyInstaller specification for server
    \item \texttt{build\_client.spec}: PyInstaller specification for client
    \item \texttt{build\_server.bat/sh}: Automated build scripts for server
    \item \texttt{build\_client.bat/sh}: Automated build scripts for client
\end{itemize}

\textbf{Startup Scripts}:
\begin{itemize}
    \item \texttt{start\_server.bat/sh}: Convenient server launcher
    \item \texttt{start\_client.bat/sh}: Convenient client launcher
\end{itemize}

\textbf{Dependencies}:
\begin{itemize}
    \item \texttt{requirements.txt}: Python package dependencies with version constraints
\end{itemize}

\subsection{5. Documentation}

\textbf{Technical Documentation} (this document): \texttt{docs/technical\_documentation.tex}
\begin{itemize}
    \item System architecture and design rationale
    \item Detailed communication protocol specifications
    \item Implementation details for all five modules
    \item Session management and threading model
    \item Platform support and compatibility
    \item Complete user guide for setup and usage
\end{itemize}

\textbf{Additional Documentation} in \texttt{docs/}:
\begin{itemize}
    \item \texttt{ARCHITECTURE.md}: System architecture diagrams and flow charts
    \item \texttt{MODULE1\_SUMMARY.md}: Video conferencing implementation
    \item \texttt{MODULE2\_SUMMARY.md}: Audio conferencing implementation
    \item \texttt{MODULE3\_SCREEN\_SHARING.md}: Screen sharing details
    \item \texttt{MODULE4\_GROUP\_CHAT.md}: Chat system implementation
    \item \texttt{MODULE5\_FILE\_SHARING.md}: File transfer system
    \item \texttt{INSTALLATION\_AND\_TESTING.md}: Setup and testing procedures
    \item \texttt{QUICKSTART.md}: Quick reference guide
\end{itemize}

\subsection{6. Build Artifacts}

When built using PyInstaller, the project generates:

\textbf{Server Executable}:
\begin{itemize}
    \item \texttt{dist/server.exe} (Windows) or \texttt{dist/server} (Linux/macOS)
    \item Standalone, no Python installation required
    \item Includes all dependencies bundled
    \item Size: ~150-200 MB
\end{itemize}

\textbf{Client Executable}:
\begin{itemize}
    \item \texttt{dist/client.exe} (Windows) or \texttt{dist/client} (Linux/macOS)
    \item Standalone, no Python installation required
    \item Includes PyQt6 GUI framework and all media libraries
    \item Size: ~200-250 MB
\end{itemize}

\section{Conclusion}

This LAN Multi-User Communication Application provides a comprehensive platform for real-time collaboration over local networks. By leveraging both TCP and UDP protocols, it delivers reliable control messaging alongside low-latency media streaming. The modular architecture allows for easy extension and customization, while the PyQt6-based user interface makes it accessible to users of all technical levels.

\subsection{Key Implementation Highlights}

\subsubsection{Networking}
\begin{itemize}
    \item \textbf{Hybrid TCP/UDP Architecture}: Uses TCP for reliable control messages and file transfers, UDP for low-latency media streaming
    \item \textbf{Multi-Port Design}: Separates different services across ports 5000-5005 for clear protocol separation
    \item \textbf{Server-Side Mixing}: Audio mixing on server prevents echo and reduces client-side processing
    \item \textbf{Thread-Safe Broadcasting}: Uses locks (\texttt{clients\_lock}, \texttt{frames\_lock}, etc.) for concurrent access to shared data
\end{itemize}

\subsubsection{Media Processing}
\begin{itemize}
    \item \textbf{JPEG Compression}: Balances quality and bandwidth with 60\% for video, 50\% for screen sharing
    \item \textbf{Noise Gate}: Client-side RMS threshold (> 100) prevents echo by filtering quiet audio
    \item \textbf{Jitter Buffer}: Queue-based audio buffering (20 frames max, 5 frame prebuffer) for smooth playback
    \item \textbf{Dynamic Packet Sizing}: Screen sharing automatically reduces quality if packets exceed UDP limits
\end{itemize}

\subsubsection{Concurrency}
\begin{itemize}
    \item \textbf{Server Threads}: 6 main threads (TCP acceptor, UDP video, UDP audio, screen control, screen UDP, file transfer) plus per-client handlers
    \item \textbf{Client Threads}: 10+ threads for capture, transmission, reception, playback, and GUI updates
    \item \textbf{PyQt6 Signals}: Thread-safe GUI updates using Qt's signal/slot mechanism
    \item \textbf{Daemon Threads}: Background threads automatically terminate when main thread exits
\end{itemize}

\subsubsection{User Experience}
\begin{itemize}
    \item \textbf{Auto Layout}: Grid automatically adjusts (1x1 to 4x4) based on participant count
    \item \textbf{Desktop Notifications}: Bottom-right popups for chat messages, user joins/leaves, file uploads
    \item \textbf{Private Messaging}: Multi-recipient selection with visual indicators
    \item \textbf{Material Design Icons}: Modern icon set using QtAwesome (with emoji fallback)
    \item \textbf{Responsive Panels}: Smooth slide-in/out animations for side panels
\end{itemize}

\subsection{Networking Concepts Demonstrated}

The application demonstrates practical implementation of:
\begin{itemize}
    \item \textbf{Client-Server Architecture}: Star topology with centralized control
    \item \textbf{TCP Reliability}: Connection management, heartbeat mechanism, message framing
    \item \textbf{UDP Real-Time Streaming}: Low-latency video/audio with acceptable packet loss
    \item \textbf{Multi-Threading}: Concurrent socket I/O without blocking
    \item \textbf{Protocol Design}: Custom binary and text-based protocols over sockets
    \item \textbf{Network Programming}: Socket creation, binding, listening, accepting, connecting
    \item \textbf{Data Serialization}: Pickle for structured data, struct for binary packing
    \item \textbf{Resource Management}: Proper socket cleanup, connection pooling, memory management
\end{itemize}

\subsection{Performance Characteristics}

\textbf{Latency}:
\begin{itemize}
    \item Video: ~100-200ms (capture + compression + network + decode + display)
    \item Audio: ~50-100ms (capture + network + mixing + playback)
    \item Screen: ~200-300ms (lower FPS acceptable for screen sharing)
\end{itemize}

\textbf{Bandwidth (per user)}:
\begin{itemize}
    \item Upstream: ~13 Mbps (12 Mbps video + 0.7 Mbps audio + overhead)
    \item Downstream: ~13 Mbps × (n-1) users (server broadcasts to all)
\end{itemize}

\textbf{Scalability}:
\begin{itemize}
    \item Current limit: 10 users (configurable in \texttt{MAX\_USERS})
    \item Server bottleneck: CPU for mixing, memory for buffering, bandwidth for broadcasting
    \item Potential improvements: Use multicast for video, optimize mixing with SIMD, implement SFU architecture
\end{itemize}

\subsection{Future Enhancements}

Potential extensions include:
\begin{itemize}
    \item \textbf{Recording}: Save conference video/audio to file
    \item \textbf{Virtual Backgrounds}: Background blur or replacement using computer vision
    \item \textbf{Encryption}: TLS for TCP, DTLS for UDP to secure communication
    \item \textbf{Scalability}: Selective Forwarding Unit (SFU) architecture for larger conferences
    \item \textbf{Codec Upgrade}: H.264/VP8 video, Opus audio for better compression
    \item \textbf{Quality Adaptation}: Automatic bitrate adjustment based on network conditions
    \item \textbf{Screen Annotation}: Drawing tools for presenter during screen sharing
    \item \textbf{Breakout Rooms}: Multiple sub-conferences within same server
\end{itemize}

\subsection{Academic Value}

This project serves as an excellent learning platform for:
\begin{itemize}
    \item Understanding client-server networking fundamentals
    \item Practical experience with TCP and UDP protocols
    \item Real-time systems programming and multi-threading
    \item Media processing and compression techniques
    \item GUI development with modern frameworks
    \item Software architecture and design patterns
\end{itemize}

The clean separation of concerns (network layer, media layer, GUI layer) makes it suitable for educational use and further experimentation.

\end{document}
